\documentclass{article}

% Language setting
% Replace `english' with e.g. `spanish' to change the document language
\usepackage[english]{babel}

% Set page size and margins
% Replace `letterpaper' with `a4paper' for UK/EU standard size
\usepackage[letterpaper,top=2cm,bottom=2cm,left=3cm,right=3cm,marginparwidth=1.75cm]{geometry}

% Useful packages
\usepackage{amsmath}
\usepackage{graphicx}
\usepackage[colorlinks=true, allcolors=blue]{hyperref}

\title{Report for WIDS: Gambling with RL}
\author{Dwija Patel}
\date{February 2025}

\begin{document}
\maketitle


\section{Introduction}

This project has given me insights into the world of reinforcement learning(RL). Over the course of this project I have learned a variety of things which can be applied in real-world problems.
The following report outlines the same.

\section{Methodology (Weekly Report)}

\subsection{Week 0}

In this week, I learned the basics of Python through the resources provided. I became familiar with the easy syntax of Python. I also explored the numpy and the matplotlib libraries of Python. I also learned how to use jupyter notebook. Using this, I completed the assignment which was a basic application of all the things I learned.

Assignment : \href{https://github.com/dwijaap/WIDS-UID74/blob/main/Assignment0.py}{Week 0}

\subsection{Week 1}

In this week, I learned about MDPs(Markov Decision Process) and how to construct MDPs for given RL problems. I learned about the bandit walk, slipper bandit walk and the frozen lake environment for which I generated MDPs using Python (in Jupyter Notebook).

Assignment : \href{https://github.com/dwijaap/WIDS-UID74/blob/main/Assignment1.ipynb}{Week 1}

\subsection{Week 2}

Week 2 was about using different strategies to maximize the reward for a given MDP. It taught me methods to come up with the best possible policy (a map which maps each state to the action with the highest state-value function in order to get maximize the reward). I learned new terms like the value function, policy-improvement, policy-iteration and value-iteration. Policy iteration and value iteration are two different methods which can generate the best possible policy for a given MDP so as to maximize the reward. I wrote the code to implement the two methods on the frozen lake MDP.

Assignment : \href{https://github.com/dwijaap/WIDS-UID74/blob/main/Assignment2.ipynb}{Week 2}

\subsection{Week 3}

Week 3 dove deeper into the world of RL and taught many techniques that can be employed in case of Multi-Armed Bandits. Multi-Armed Bandits were introduced which are basically bandits which have multiple arms i.e. they are MDPs with with a single non-terminal state and a single time-step per episode. Now, to solve these MABs, two different approaches were applied : random exploration strategies and optimistic exploration strategies. As the names suggest, the first one employs an approach where we randomly explore all the actions and at the end the state-value function of all actions converge to the real reward function for a given action. The second one employs exploration as well as exploitation i.e. along with exploration, it also selects the actions with a higher reward function while iteration. According to these strategies, I implemented different strategies such as : epsilon-greedy strategy, UCB, KL-UCB, and Thompson Sampling. The latter 3 are based on optimistic exploration whereas the first one is based on the random exploration strategy.

Assignment : \href{https://github.com/dwijaap/WIDS-UID74/blob/main/Assignment3.ipynb}{Week 3}


\subsection{Week 4}

This was the most interesting week and this week included the application of all our learnings to a real-life problem. In this week, I constructed a program for the game Black Jack. The code written implements TD($\lambda$) to get the state-value function for each of the observation tuples. TD($\lambda$) is mid way between Temporal difference and Monte Carlo (MC) prediction. 

Assignment : \href{https://github.com/dwijaap/WIDS-UID74/blob/main/Black-Jack%20Project.ipynb}{Black-Jack projects}

\section{Conclusion}

This project has given me a lot of insight into the world of Reinforcement Learning and how different strategies in RL can be employed to solve real-world problems. One of them that we solved was the Black-Jack game. In all, it was a fun project.

\end{document}